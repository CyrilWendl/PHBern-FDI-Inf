%% mathematics %%
% left ( and right ) round brackets
\newcommand{\lr}[1]{\left(#1\right)}
% left { and right } curly brackets
\newcommand{\lrc}[1]{\left\{#1\right\}}
% left [ and right ] square brackets
\newcommand{\lrs}[1]{\left[#1\right]}
% N (set of natural numbers)
\newcommand{\N}{\mathbb{N}}
% set of units in N = N\{0}
\newcommand{\Nunit}{\N^{\times}}
% Z (set of integers)
\newcommand{\Z}{\mathbb{Z}}
% set of units in Z = Z\{0}
\newcommand{\Zunit}{\Z^{\times}}
% Q (set of rationals)
\newcommand{\Q}{\mathbb{Q}}
% R (set of reals)
\newcommand{\R}{\mathbb{R}}
% set of units in R = R\{0}
\newcommand{\Runit}{\R^{\times}}
% R >= 0 (non negative real numbers)
\newcommand{\Rnn}{\R_{\geq 0}}
% R > 0 (positive real numbers)
\newcommand{\Rp}{\R_{>0}}
% C (set of complex numbers)
\newcommand{\C}{\mathbb{C}}
% set of units in C = C\{0}
\newcommand{\Cunit}{\C^{\times}}
% arg
\newcommand{\Carg}[1]{\text{arg}\lr{#1}}
% cis
\newcommand{\cis}[1]{\text{cis}\lr{#1}}
% K
\newcommand{\K}{\mathbb{K}}
% set of units in K = K\{0}
\newcommand{\Kunit}{\K^{\times}}
% set of units
\newcommand{\unitset}[1]{#1^{\times}}
% real part of a complex number Re(...)
\newcommand{\real}[1]{\text{Re}\lr{#1}}
% imaginary part of a complex number Im(...)
\newcommand{\imag}[1]{\text{Im}\lr{#1}}
% absolute value
\newcommand{\abs}[1]{\left\lvert#1\right\rvert}
% norm
\newcommand{\norm}[1]{\left\lVert#1\right\rVert}
% closed interval [a,b]
\newcommand{\ic}[2]{\lrs{#1, #2}}
% closed / open interval [a,b)
\newcommand{\ico}[2]{\left[#1, #2\right)}
% open / closed interval (a,b]
\newcommand{\ioc}[2]{\left(#1, #2\right]}
% open interval (a,b)
\newcommand{\io}[2]{\lr{#1, #2}}
% double fraction
\newcommand{\doublefraction}[4]{\cfrac{\frac{#1}{#2}}{\frac{#3}{#4}}}
% symbol :=
\newcommand{\defeql}{\mathrel{\mathop:}=}
% symbol =:
\newcommand{\defeqr}{=\mathrel{\mathop:}}
% ceiling operator
\newcommand{\ceil}[1]{\left\lceil #1 \right\rceil}
% floor operator
\newcommand{\floor}[1]{\left\lfloor #1 \right\rfloor}
% text italics and bold font
\newcommand{\tib}[1]{\textbf{\textit{#1}}}
% set (condition is in math mode): {n\in\N ; n < 10}
\newcommand{\setcm}[2]{\lrc{\;#1 \; ; \; #2 \;}}
% set (condition is in text mode): {n\in\N ; $n$ is even}
\newcommand{\setct}[2]{\lrc{\;#1 \; ; \; \text{#2} \;}}
% set (all in text mode): {$n$ is a natural number ; $n$ is even}
\newcommand{\settt}[2]{\lrc{\; \text{#1} \; ; \; \text{#2} \;}}
% replace this (not necessary, but used in legacy code)
\newcommand{\elem}[1]{\xrightarrow{\substack{#1}}}
% roman numerals
\newcommand{\romanNumeral}[1]{%
  \textup{\uppercase\expandafter{\romannumeral#1}}%
}
%% some colors %%
\newcommand{\red}[1]{\textcolor{Red}{#1}}
\newcommand{\green}[1]{\textcolor{ForestGreen}{#1}}
\newcommand{\blue}[1]{\textcolor{Blue}{#1}}
\newcommand{\orange}[1]{\textcolor{Orange}{#1}}

%% code / programming %%
% Python inline
\newcommand{\pythoninline}[1]{{\lstinline[language=Python,basicstyle=\ttfamily\normalsize]{#1}}}
% TigerJython inline
\newcommand{\tjinline}[1]{{\lstinline[language=Python,style=TigerJython,basicstyle=\ttfamily\normalsize]{#1}}}
% C++ inline
\newcommand{\cppinline}[1]{{\lstinline[language=C++,basicstyle=\ttfamily\normalsize]{#1}}}

\pgfset{foreach/parallel foreach/.style args={#1in#2via#3}{evaluate=#3 as #1 using {{#2}[#3-1]}}}

% macro for creating a title page
\newcommand{\maketitlepage}[2]{
  % arg1: title
  % arg2: subtitle
  \begin{titlepage}
    \vspace*{0.25\textheight}
    \begin{center}

      % logo
      %	      \begin{figure}[H]
      %	        \centering
      %	        \includegraphics[height=.15\textwidth]{LeeLogo}
      %	      \end{figure}
      \textbf{Informatik}\\
      % title
      \newcommand{\HRule}{\rule{\linewidth}{0.2mm}}
      \HRule \\[0.5cm]
      \ifblank{#2}{
      % if no subtitle is given (#2 blank)
      {\huge #1}\\[0.5cm]

      \begin{tikzpicture}[remember picture,overlay]
        % logo
        % Define the desired (x, y) coordinate	          
        % Create a scope for the centered image
        % Insert your external image
        \node[draw=white!10,fill=white!20, opacity=0.3] at (0,4) {\includegraphics[width=.5\textwidth]{LeeLogo}};


        \begin{scope}[on background layer]
          \begin{scope}[opacity=0.25]
            % Initialize a scaling factor for the emojis
            \newcommand{\scaleFactor}{4}
            \newcommand{\spacing}{0.3} % smaller number = emojis closer together
            \newcommand{\numberEmojis}{70}
            \newcommand{\rotateAngle}{10}
            \newcommand{\numberSpirals}{7}
            \def\arrayA{1,2,3,4,5,6,7}

            \foreach \x [count=\c,
              parallel foreach=\nSpiral in \arrayA via \c,
              parallel foreach=\theemoji in \arrayB via \c]
            in \arrayA
            {
              % calculate the starting angle for each spiral
              \pgfmathsetmacro{\startAngle}{\nSpiral/\numberSpirals*360}
              \foreach \i in {0,1,...,\numberEmojis} { % Increase the number for a larger spiral
                  % Calculate the angle, radius, and scale for each emoji
                  \pgfmathsetmacro{\angle}{\startAngle + \i * \rotateAngle}
                  \pgfmathsetmacro{\radius}{\spacing * \i}
                  \pgfmathsetmacro{\scale}{\scaleFactor * (\i/\numberEmojis)} % Adjust the divisor for size variation
                  % Increase spacing gradually
                  \pgfmathsetmacro{\spacing}{\spacing + \i/\numberEmojis} % Adjust the divisor for spacing variation
                  % Select the emoji to be placed in the spiral
                  \node[scale=\scale] at (\angle:\radius) {\emoji{\theemoji}};
                }
            }
          \end{scope}
        \end{scope}
      \end{tikzpicture}
      {\large Skript}\\[0.2cm]
      }{
      % if #2 not blank
      {\huge #1}\\[0.5cm]
      {\large #2}\\[0.2cm]
      }

      \HRule \\[0.5cm]

      {\large \textcopyleft{} Winterthur, \today}

      %      \tikz[remember picture,overlay] \node[opacity=0.3,inner sep=0pt] at (current page.center){\includegraphics[width=\paperwidth,height=\paperheight]{LeeLogo}};

    \end{center}
  \end{titlepage}
}

\NewTotalTCBox{\commandbox}{ s v } {verbatim,colupper=white,colback=black!75!white,colframe=black} {\IfBooleanT{#1}{\textcolor{red}{\ttfamily\bfseries > }}%
  \lstinline[language=command.com,keywordstyle=\color{blue!35!white}\bfseries]^#2^}


% Boxen für Beamer-Aufgaben, -Abgaben, -Challenges und -Aufträge, in ähnlichem Farschema wie Skript
\NewTotalTCBox{\abgabebox}{ s m } {on line,colupper=white,colback=Purple,colframe=Purple, size=title} {\faArrowUp{} #2}

\NewTotalTCBox{\aufgabebox}{ s m } {on line,colupper=white,colback=RoyalBlue,colframe=RoyalBlue, size=title} {\faPencilSquareO{} #2}


\NewTotalTCBox{\challengebox}{ s m } {on line,colupper=white,colback=Gold!80!black,colframe=Gold!80!black, size=title} {\faTrophy{} Challenge: #2}

\NewTotalTCBox{\readbox}{ s m } {on line,colupper=white,colback=ForestGreen,colframe=ForestGreen, size=title} {\faBook{} Lesen Sie Abschnitt #2}

\nottoggle{beamer}{
% To-Do-Listen, z.B. für Lernziele
\newlist{todolist}{itemize}{2}
\setlist[todolist]{label=$\square$}}{}
