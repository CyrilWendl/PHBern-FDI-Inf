%% settings for listings-package %%
% basline listings style
\lstdefinestyle{BaselineStyle}
{ %
  commentstyle=\color{Blue},
  keywordstyle=\color{Green},
  numberstyle=\tiny\color{Gray},
  stringstyle=\color{Fuchsia},
  basicstyle=\ttfamily,
  columns=flexible,
  breakatwhitespace=false,
  breaklines=true,
  captionpos=b,
  keepspaces=true,
  numbers=left,
  numbersep=5pt,
  showspaces=false,
  showstringspaces=false,
  showtabs=false,
  tabsize=2,
  xleftmargin=1.5em,
  frame=single,
  framexleftmargin=1.5em,
  autogobble=true
} %
% set to 'BaselineStyle'
\lstset{style=BaselineStyle}

% TigerJython syntax
\lstdefinestyle{TigerJython}
{ %
  language=Python,
  morekeywords={repeat, sqrt},
  backgroundcolor=\color{Pink!15},
  frame=none,
  framexleftmargin=0em,
  xleftmargin=0em,
  belowskip=0pt
} %

\lstdefinestyle{sql}
{ %
  %	backgroundcolor=\color{BlanchedAlmond!20},
  language=sql,
  frame=lines,
  numbers=none,
  deletekeywords={ROLE},
  morekeywords={LENGTH,IS,RENAME,TO,REFERENCES,ENUM,TINYINT,DATETIME,AUTO_INCREMENT},
  linewidth=\linewidth,
  framexleftmargin=0em,
  xleftmargin=0em,
  belowskip=0pt
} %

% to show a LaTeX source code and its compiled resulting document in a tcolorbox
\tcbuselibrary{listings}

%% specific settings for different document classes %%
\iftoggle{exam}{
  %% EXAM ENVIRONMENT %%
  \iftoggle{german}{
    \pointpoints{Punkt}{Punkte}
    \bonuspointpoints{Bonuspunkt}{Bonuspunkte}
    \renewcommand{\solutiontitle}{\noindent\textbf{Lösungsvorschlag:}\enspace}
    \hqword{Aufgabe:}
    \hpgword{Seite:}
    \hpword{Punkte:}
    \hsword{erhalten:}
    \htword{Total}
    \bhqword{Aufgabe:}
    \bhpgword{Seite:}
    \bhpword{Bonuspunkte:}
    \bhsword{erhalten:}
    \bhtword{Total}
    \chqword{Aufgabe:}
    \chpgword{Seite:}
    \chpword{Punkte:}
    \chbpword{Bonuspunkte:}
    \chsword{erhalten:}
    \chtword{Total}
    \totalformat{Aufgabe \thequestion\ total: \totalpoints\ Punkte}
  }{}
  %\checkedchar{\CheckedBox}
  \runningheadrule
  % \firstpageheader{\mycourse}{\myclass}{\mydate}
  \runningheader{\mycourse}{\myclass}{\mydate}
  \firstpagefooter{}{\thepage\,/\,\numpages}{}
  \runningfooter{}{\thepage\,/\,\numpages}{}
  \CorrectChoiceEmphasis{\normalfont}
  % \renewcommand{\questionlabel}{\bfseries Aufgabe~\thequestion.}
}{}

\iftoggle{book}{
  % fancyheader options
  \pagestyle{fancy}
  \fancyhead[C]{}
  \lhead[ \leftmark ]{\thetopic}
  % \chead{\nouppercase\leftmark}
  \rhead[Informatik]{\textcopyleft{} KS Im Lee, Informatik, \the\year}
  \setlength{\headheight}{14pt}
  
  % BOOK ENVIRONMENT %
  \setcounter{secnumdepth}{5}
  \setcounter{tocdepth}{5}
  % start chapter numbering at different number (set to -1 to start at 0)
  \setcounter{chapter}{0}
  % slightly changing chapter marking in header
  \renewcommand{\chaptermark}[1]{\markboth{\textnormal{\thechapter}\ \textnormal{#1}}{}}
  % slightly changing section marking in header
  \renewcommand{\sectionmark}[1]{\markright{\textnormal{\thesection}\ \textnormal{#1}}{}}
  % making header of toc non italics (\upshape) and not all capitals
  \addto\captionsgerman{\renewcommand{\contentsname}{\upshape{I\MakeLowercase{nhaltsverzeichnis}}}}
}{}

% BEAMER %
\iftoggle{beamer}{
  \beamertemplatenavigationsymbolsempty
}{}

\iftoggle{german}{
  %% replace English terms with German counterparts %%
  \renewcommand{\lstlistingname}{Programm}
  \floatname{algorithm}{Algorithmus}
  \renewcommand{\algorithmicrequire}{\textbf{Eingabe:}}
  \renewcommand{\algorithmicensure}{\textbf{Resultat:}}

  %% clever refs %%
  \nottoggle{beamer}{
    \crefname{lstlisting}{Programm}{Programme}
    \Crefname{lstlisting}{Programm}{Programme}
  }{}
}{}

 \iftoggle{article}{
 	\iftoggle{addheader}{
 	
   % fancyheader options
   \pagestyle{fancy}
   \fancyhead[C]{}
   \lhead[ \leftmark ]{\leftheadertext}
   % \chead{\nouppercase\leftmark}
   \rhead[Informatik]{\rightheadertext}
   \setlength{\headheight}{14pt}}{}
 }{}
