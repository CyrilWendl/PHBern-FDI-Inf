\chapter{Beispiele verschiedener Umgebungen}

\section{Aufgaben und Antworten}
% \begin{tcblisting}{colback=red!5!white,
% colframe=red!75!black,title=\texttt{aufgabe}-Umgebung mit \texttt{antwort}-Umgebung, 
% fonttitle=\bfseries, 
% listing side text}
% \begin{myexercise}[Polynomfunktion]
% \label{myexercise:evalPoly}
%   Gegeben sei die Polynomfunktion
%   \begin{align*}
%     p:  \R &\to \R, \\
%         x  &\mapsto x^2.
%     \end{align*}
%   Bestimmen Sie die Zahl $p(3)$.
%     \begin{myanswer}
%       Die gesuchte Zahl ist gegeben durch
%       \begin{align*}
%         p(3) = 3^2 = 9.    
%       \end{align*}
%     \end{myanswer}
% \end{myexercise}
% In \cref{myexercise:evalPoly} haben wir gesehen, wie eine Polynomfunktion ausgewertet werden kann.
% \end{tcblisting}
% Das Setzen eines Titels ist optional. Falls die Aufgabe keinen Titel trägt, lässt man die eckigen Klammern einfach vollständig weg (nicht
% einmal leere eckige Klammern).

% \begin{tcblisting}{colback=red!5!white,
% colframe=red!75!black,title=\texttt{aufgabe}-Umgebung mit \texttt{myantwort}-Umgebung, 
% fonttitle=\bfseries, 
% listing side text}
% \begin{aufgabe}{}
% Geben Sie alle Benutzernamen in sortierter Reihenfolge aus (a $\rightarrow$ z).
% \begin{myantwort}
% \begin{lstlisting}[style=sql]
% SELECT username
% FROM users
% ORDER BY username ASC
% \end{lstlisting}
% \end{myantwort}
% \end{aufgabe}
% \shipoutAnswer
% \end{tcblisting}
%

\begin{myexercise}[Polynomfunktion]
\label{myexercise:evalPoly}
  Gegeben sei die Polynomfunktion
  \begin{align*}
    p:  \R &\to \R, \\
        x  &\mapsto x^2.
    \end{align*}
  Bestimmen Sie die Zahl $p(3)$.
    \begin{myanswer}
      Die gesuchte Zahl ist gegeben durch
      \begin{align*}
        p(3) = 3^2 = 9.    
      \end{align*}
    \end{myanswer}
\end{myexercise}

\begin{mysubmission}[Polynomfunktion]
\label{mysubmission:evalPoly}
  Gegeben sei die Polynomfunktion
  \begin{align*}
    p:  \R &\to \R, \\
        x  &\mapsto x^2.
    \end{align*}
  Bestimmen Sie die Zahl $p(3)$.
    \begin{myanswer}
      Die gesuchte Zahl ist gegeben durch
      \begin{align*}
        p(3) = 3^2 = 9.    
      \end{align*}
    \end{myanswer}
\end{mysubmission}

\begin{mychallenge}[Polynomfunktion]
\label{mychallenge:evalPoly}
  Gegeben sei die Polynomfunktion
  \begin{align*}
    p:  \R &\to \R, \\
        x  &\mapsto x^2.
    \end{align*}
  Bestimmen Sie die Zahl $p(3)$.
    \begin{myanswer}
      Die gesuchte Zahl ist gegeben durch
      \begin{align*}
        p(3) = 3^2 = 9.    
      \end{align*}
    \end{myanswer}
\end{mychallenge}
In \cref{myexercise:evalPoly}, \cref{mysubmission:evalPoly} und \cref{mychallenge:evalPoly} haben wir gesehen, wie eine
Polynomfunktion ausgewertet werden kann. Der Titel in eckigen Klammern ist optional und kann jeweils weggelassen werden. Dann
ist es aber schön (nicht zwingend), wenn auch die leeren eckigen Klammern nicht geschrieben werden:
\begin{mysubmission}
  Gegeben sei die Polynomfunktion
  \begin{align*}
    p:  \R &\to \R, \\
        x  &\mapsto x^2.
    \end{align*}
  Bestimmen Sie die Zahl $p(3)$.
    \begin{myanswer}
      Die gesuchte Zahl ist gegeben durch
      \begin{align*}
        p(3) = 3^2 = 9.    
      \end{align*}
    \end{myanswer}
\end{mysubmission}

\section{Hinweisboxen}
\begin{mycaution}
  Hier steht etwas extrem wichtiges! 
\end{mycaution}

\section{amsthm}
Alle anderen Environments verwenden direkt das \texttt{amsthm}-Paket\dots

\begin{myexample}\label{myexample:01}
  Super schönes Beispiel. \lipsum[1]
\end{myexample}

In \cref{myexample:01} haben wir tollen Text gesehen.

\begin{mydefinition}
  Eine neue Definition.
\end{mydefinition}

\begin{myremark}
  Eine Bemerkung.
\end{myremark}
